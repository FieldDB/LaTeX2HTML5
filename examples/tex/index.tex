\begin{pspicture}(0,-3)(8,3)
\rput(0,0){$x(t)$}
\rput(4,1.5){$f(t)$}
\rput(4,-1.5){$g(t)$}
\rput(8.2,0){$y(t)$}
\rput(1.5,-2){$h(t)$}
\psframe(1,-2.5)(7,2.5)
\psframe(3,1)(5,2)
\psframe(3,-1)(5,-2)
\rput(4,0){$X_k = \frac{1}{p} \sum \limits_{n=\langle p\rangle}x(n)e^{-ik\omega_0n}$}
\psline{->}(0.5,0)(1.5,0)
\psline{->}(1.5,1.5)(3,1.5)
\psline{->}(1.5,-1.5)(3,-1.5)
\psline{->}(6.5,1.5)(6.5,0.25)
\psline{->}(6.5,-1.5)(6.5,-0.25)
\psline{->}(6.75,0)(7.75,0)
\psline(1.5,-1.5)(1.5,1.5)
\psline(5,1.5)(6.5,1.5)
\psline(5,-1.5)(6.5,-1.5)
\psline(6,-1.5)(6.5,-1.5)
\pscircle(6.5,0){0.25}
\psline(6.25,0)(6.75,0)
\psline(6.5,0.5)(6.5,-0.5)
\end{pspicture}

Many of us think our thoughts using a language of some sort---there is usually some voice in our minds. Language in some ways, makes us who we are. Some even argue in the world of cognitive science that language is the foundation of our consciousness. 


An author who has in their minds representations of intelligent concepts should be able to freely express herself through language with free association---digital expressions of these ideas in some cases requires total control of the computer and all of its processes. 


The vision behind the personal computer was that any person could have full command of the functions of their device. I think this vision has come true to some degree, but not fully when it comes to creating graphics, especially mathematical diagrams online. 


Does the common mathematician or professor have the ability to express concepts through web technology? The Web has its own language, and the goal of this project is to help blur the lines between what authoring the mathematical Web should be like and typesetting beautiful Math.


If you know \LaTeX, then get ready to author interactive diagrams in real-time (try using mouse or touch to interact with diagrams).


\begin{interactive}

\waypoint{Lines And Vectors}

What matters most is minimizing the distance between our expression of an idea and the execution of that idea. For example, I can describe a vector at $(0,0)$ and initial value of the head at $(2,2)$ that will follow a user touch or mouse event. This will produce the following interaction:


\begin{center}
\begin{pspicture}(-2,-2)(2,2)
\psframe(-2,-2)(2,2)
\userline[linewidth=1.5 pt]{->}(0,0)(2,2)
\end{pspicture}
\end{center}


This was as easy as using this \TeX, which many math professors could understand.

\begin{verbatim}
\begin{pspicture}(-2,-2)(2,2)
\psframe(-2,-2)(2,2)
\userline[linewidth=1.5 pt]{->}(0,0)(2,2)
\end{pspicture}
\end{verbatim}

If you specify more arguments, you can create functions for the head and and tail of the vector, which each takes the current $x$ and $y$ position of the users finger or cursor as they move and produces the following interaction:

\begin{interactive}

\begin{center}
\begin{pspicture}(-2,-2)(2,2)
\psframe(-2,-2)(2,2)
\userline[linewidth=2pt,linecolor=green]{->}(0,0)(2,2){-x}{-y}
\userline[linewidth=2pt,linecolor=red]{->}(0,0)(2,2){0}{y}
\userline[linewidth=2pt,linecolor=purple]{->}(0,0)(2,2){-x}{cos(y)}
\userline[linewidth=2pt,linecolor=lightblue]{->}(0,0)(2,2)(sin(x)}{-y}
\end{pspicture}
\end{center}

\end{interactive}


2 extra arguments provide functions for the head, 4 extra arguments allows you to control both and tail

\begin{verbatim}
\userline[linewidth=2pt,linecolor=green]{->}(0,0)(2,2){-x}{-y}
\userline[linewidth=2pt,linecolor=red]{->}(0,0)(2,2){0}{y}
\userline[linewidth=2pt,linecolor=purple]{->}(0,0)(2,2){-x}{cos(y)}
\userline[linewidth=2pt,linecolor=lightblue]{->}(0,0)(2,2)(sin(x)}{-y}
\end{verbatim}

I can also draw a more complex version, and start to make more useful diagrams to describe vectors:

\begin{center}
\begin{pspicture}(-5,-5)(5,5)
        
% y-axis
\rput(0.3,3.75){ $Im$ }
\psline{->}(0,-3.75)(0,3.75)

% x-axis
\rput(3.75,0.3){ $Re$ }
\psline{->}(-3.75,0)(3.75,0)
            
% the circle
\pscircle(0,0){ 3 }
            

 % new vector 
\rput(2.3,1){$e^{i\omega}-\alpha$}
\userline[linewidth=1.5 pt]{->}(1.500,0.000)(2.121,2.121)
\userline[linewidth=1.5 pt,linecolor=blue]{->}(0,0.000)(2.121,2.121){(x>0) ? 3 * cos( atan(-y/x) ) : -3 * cos( atan(-y/x) ) }{ (x>0) ? -3 * sin( atan(-y/x) ) : 3 * sin( atan(-y/x) )}

\userline[linewidth=1.5 pt,linestyle=dashed](-1.500,0.000)(2.121,2.121){x}{0}{x}{y}
\userline[linewidth=1.5 pt,linestyle=dashed](-1.500,0.000)(2.121,2.121){0}{y}{x}{y}

\rput(-0.75,-4.25){$1+\alpha$}
\rput(2.25,-4.25){$1-\alpha$}
\psline{<->}(-3,-4)(1.5,-4)
\psline{<->}(1.5,-4)(3,-4)
\psline[linestyle=dashed](3,-4.5)(3,0)
\psline[linestyle=dashed](-3,-4.5)(-3,0)
\psline[linestyle=dashed](1.5,-4.5)(1.5,0)

    
\end{pspicture}
\end{center}


\begin{verbatim}
\begin{pspicture}(-5,-5)(5,5)
\rput(0.3,3.75){ $Im$ }
\psline{->}(0,-3.75)(0,3.75)
\rput(3.75,0.3){ $Re$ }
\psline{->}(-3.75,0)(3.75,0)
\pscircle(0,0){ 3 }
\rput(2.3,1){$e^{i\omega}-\alpha$}
\userline[linewidth=1.5 pt]{->}(1.500,0.000)(2.121,2.121)
\userline[linewidth=1.5 pt,linecolor=blue]{->}(0,0.000)(2.121,2.121){(x>0) ? 3 * cos( atan(-y/x) ) : -3 * cos( atan(-y/x) ) }{ (x>0) ? -3 * sin( atan(-y/x) ) : 3 * sin( atan(-y/x) )}
\userline[linewidth=1.5 pt,linestyle=dashed](-1.500,0.000)(2.121,2.121){x}{0}{x}{y}
\userline[linewidth=1.5 pt,linestyle=dashed](-1.500,0.000)(2.121,2.121){0}{y}{x}{y}
\rput(-0.75,-4.25){$1+\alpha$}
\rput(2.25,-4.25){$1-\alpha$}
\psline{<->}(-3,-4)(1.5,-4)
\psline{<->}(1.5,-4)(3,-4)
\psline[linestyle=dashed](3,-4.5)(3,0)
\psline[linestyle=dashed](-3,-4.5)(-3,0)
\psline[linestyle=dashed](1.5,-4.5)(1.5,0)
\end{pspicture}
\end{verbatim}

\waypoint{User Variables}


The {\tt uservariable} command represents a variable that changes with user interaction such as touch or mouse events over the graphic.

\begin{verbatim}
\uservariable{variable}(x1,y1){f(x,y)}
\end{verbatim}
{\tt variable} is the variable that you can use in {\tt psplot} and {\tt userline} parameters and equations. Eventually it should be something you can use everywhere, but for the scope of this project, we kept it limited.


{\tt x1} and {\tt y1} are the initial values of {\tt x} and {\tt y} for the function $f(x,y)$ that is that last argument for the command. If there are no user interactions, the function is evaluated with the specified inital values.


On any user event, such as a touch or mouse event, the function $f(x,y)$ is evaluated and any dependent commands, namely {\tt psplot} and {\tt userline} are immediately notified and re-rendered with updated execution contexts.

Think about the student learning a new concept. Here is an example of teaching integration using ``area under the curve'', for example, you can represent $\int_a^b f(x) dx$ as:

\begin{center}
\begin{pspicture}(-4,-3)(4,3)
        
\uservariable{alpha}(0,0){x}

\psplot[algebraic,linewidth=2pt,fillstyle=solid, fillcolor=lightblue]{-4}{alpha}{sin(x)}
\psplot[algebraic,linewidth=2pt]{-4}{4}{sin(x)}
    
\psline{->}(-4,0)(4,0)

    
\end{pspicture}
\end{center}

\begin{verbatim}
\begin{pspicture}(-4,-3)(4,3)
\uservariable{alpha}(0,0){x}
\psplot[algebraic,linewidth=2pt,fillstyle=solid, fillcolor=lightblue]{-4}{alpha}{sin(x)}
\psplot[algebraic,linewidth=2pt]{-4}{4}{sin(x)}
\psline{->}(-4,0)(4,0)
\end{pspicture}
\end{verbatim}

The next example allows you to move the height of the graph and also integrate over a moving interval.

\begin{center}
\begin{pspicture}(-4,-3)(4,3)

\uservariable{alpha}(0,0){x}
\uservariable{beta}(0,0){y}

\psplot[algebraic,linewidth=2pt,fillstyle=solid, fillcolor=lightblue]{alpha-3}{alpha}{beta + sin(x)}
\psplot[algebraic,linewidth=2pt]{-4}{4}{beta + sin(x)}

\psline{->}(-4,0)(4,0)

\end{pspicture}
\end{center}


\begin{verbatim}
\begin{pspicture}(-4,-3)(4,3)
\uservariable{alpha}(0,0){x}
\uservariable{beta}(0,0){y}
\psplot[algebraic,linewidth=2pt,fillstyle=solid, fillcolor=lightblue]{alpha-3}{alpha}{beta + sin(x)}
\psplot[algebraic,linewidth=2pt]{-4}{4}{beta + sin(x)}
\psline{->}(-4,0)(4,0)
\end{pspicture}
\end{verbatim}

\end{interactive}

\waypoint{Sliders}

One useful way to expose basic interactive functionality for an end user is the {\tt slider} command. 


Arguments for the {\tt slider} command are:

\begin{verbatim}
 \slider{min}{max}{variable}{latex}{default}
\end{verbatim}

The min and max values are for specifying the minimum and maximum of the range of the slider. 


The {\tt latex} argument is what to display next to the slider to indicate to the end user what variable the slider is changing. Finally, the most important is the {\tt variable} argument. The {\tt variable} specifies the variable that is changed based on the values of the slider, and can be used in the equations of {\tt psplot} commands. 


Note that using these ``extended'' features are not backwards compatible, a.k.a they don't work on paper!


\begin{interactive}
\psset{unit=1cm}

\begin{center}
\begin{pspicture}(-3.5,-1)(3.75,3.5)

\slider{1}{8}{n}{$N$}{4}

\psplot[algebraic,linewidth=1.5pt,plotpoints=1000]{-3.14}{3.14}{cos(n*x/2)+1.3}
\psaxes[showorigin=false,labels=none, Dx=1.62](0,0)(-3.25,0)(3.25,2.5)

\psline[linestyle=dashed](-3.14,0.3)(3.14,0.3)
\psline[linestyle=dashed](-3.14,2.3)(3.14,2.3)
\rput(3.6,2.3){$\frac{1}{1-\alpha}$}
\rput(3.6,0.3){$\frac{1}{1+\alpha}$}


\rput(3.14, -0.35){$\pi$}
\rput(1.62, -0.35){$\pi/2$}
\rput(-1.62, -0.35){$-\pi/2$}
\rput(-3.14, -0.35){$-\pi$}
\rput(0, -0.35){$0$}

\end{pspicture}
\end{center}
\end{interactive}

\begin{verbatim}
\begin{pspicture}(-3.5,-1)(3.75,3.5)
\slider{1}{8}{n}{$N$}{4}
\psplot[algebraic,linewidth=1.5pt,plotpoints=1000]{-3.14}{3.14}{cos(n*x/2)+1.3}
\psaxes[showorigin=false,labels=none, Dx=1.62](0,0)(-3.25,0)(3.25,2.5)
\psline[linestyle=dashed](-3.14,0.3)(3.14,0.3)
\psline[linestyle=dashed](-3.14,2.3)(3.14,2.3)
\rput(3.6,2.3){$\frac{1}{1-\alpha}$}
\rput(3.6,0.3){$\frac{1}{1+\alpha}$}
\rput(3.14, -0.35){$\pi$}
\rput(1.62, -0.35){$\pi/2$}
\rput(-1.62, -0.35){$-\pi/2$}
\rput(-3.14, -0.35){$-\pi$}
\rput(0, -0.35){$0$}
\end{pspicture}
\end{verbatim}



\begin{interactive}
\psset{unit=0.5cm}
\begin{center}
\begin{pspicture}(-13,-5)(13,10)

\slider{1}{8}{a}{amplitude}{4}
\slider{1}{8}{n}{frequency}{4}

\psplot[algebraic,linewidth=1.5pt,plotpoints=1000]{-12.56}{12.56}{a*sin(n*x)/(n*x)}
\psaxes[showorigin=false,labels=none, Dx=3.14](0,0)(-12.6,0)(12.6,0)

\rput(0, -0.5){$0$}

\end{pspicture}
\end{center}
\end{interactive}


\begin{verbatim}
\begin{pspicture}(-13,-5)(13,10)
\slider{1}{8}{a}{amplitude}{4}
\slider{1}{8}{n}{frequency}{4}
\psplot[algebraic,linewidth=1.5pt,plotpoints=1000]{-12.56}{12.56}{a*sin(n*x)/(n*x)}
\psaxes[showorigin=false,labels=none, Dx=3.14](0,0)(-12.6,0)(12.6,0)
\rput(0, -0.5){$0$}
\end{pspicture}
\end{verbatim}



\begin{center}
\psset{unit=0.75cm}
\begin{pspicture}(-4,-3)(4,6)
\uservariable{alpha}(0.1,0){x}
\psplot[algebraic,linewidth=2pt]{-4}{4}{pow(x,2)}
\psplot[algebraic,linecolor=blue,linewidth=3]{-4}{4}{4*(x-alpha)*alpha}
\psline{->}(-4,0)(4,0)
\end{pspicture}
\end{center}

\begin{verbatim}
\begin{pspicture}(-4,-3)(4,6)
\uservariable{alpha}(0.1,0){x}
\psplot[algebraic,linewidth=2pt]{-4}{4}{pow(x,2)}
\psplot[algebraic,linecolor=blue,linewidth=3]{-4}{4}{4*(x-alpha)*alpha}
\psline{->}(-4,0)(4,0)
\end{pspicture}
\end{verbatim}


\waypoint{math excerpt}

What frequencies can be present in a signal $\vec{x}$? The harmonics are given by a signal with period $p$ are $\{0,\omega_0,2\omega_0,\dots,(p-1)\omega_0\}$. Notice that $p\omega_0$ is missing. This is because $e^{ip\omega_0n}=e^{i2\pi n} = e^{i0n}$. Keep in mind that $\omega_0 = 2\pi/p$. In fact, the complex exponential basis of vectors is periodic with respect to $n$ and their indices $k$. 

\begin{align*}
\Psi_k(n+p) &= e^{ik\omega_0(n+p)}=e^{ik\omega_0n}e^{ik\omega_0p}=e^{ik\omega_0n} = \Psi_k(n) \\
\Psi_{k+p}(n) &= e^{i(k+p)\omega_0n} = e^{ik\omega_0n}e^{ip\omega_0n} = \Psi_k(n) 
\end{align*}

So we know that $\Psi_{-1}(n) = e^{-i\omega_0n} = \Psi_{p-1}(n) = e^{i(p-1)\omega_0n }$. So we can see that in general, we can write out a discrete fourier series expansion over any continuous set of $p$ integers. The notation used is
$$ x(n) = \sum \limits_{k=\langle p \rangle} X_ke^{ik\omega_0n}$$


Given a periodic signal $x$ in CT, we can estimate the signal using the formula: 

$$ \hat{x}(t) = \sum \limits_{k=-N}^{N}\alpha_k e^{ik\omega_0 t}   $$

How can we determine the coefficients that make the estimates closest in terms of error energy? Let $W\in\R^{2N+1}$ be a subspace spanned by the set of orthogonal basis vectors $\Psi_0, \Psi_1, \Psi_{-1} \dots, \Psi_{N}, \Psi_{-N}$. If $x$ is a vector that is not in the column space of $W$, then we can project $x$ onto the column space of $W$, producing an approximation vector $\hat{x}$, which has a distance of $\abs{\left| \mathcal{E}_N\right|}$ from the vector $x$. The vectors $\mathcal{E}_N$ and $\hat{x}$ are orthogonal.


\psset{unit=1cm}
\begin{center}
\begin{pspicture}(-1,-3)(9,4)
    
\pscustom[fillstyle=solid,fillcolor=gray!40,linestyle=none]{
    \psline[linewidth=1 pt](0,0)(4,1.2)
    \psline[linewidth=1 pt](4,1.2)(8.4,0)
    \psline[linewidth=1 pt](8.4,0)(4,-1.2)
    \psline[linewidth=1 pt](4,-1.2)(0,0)
}

\psline[linewidth=1 pt](0,0)(4,1.2)
\psline[linewidth=1 pt](4,1.2)(8.4,0)
\psline[linewidth=1 pt](8.4,0)(4,-1.2)
\psline[linewidth=1 pt](4,-1.2)(0,0)

\rput(0.78,0){$W$}


 % new vector 
\rput(6,3.3){$x$}
\psline[linewidth=1.5 pt,linecolor=red]{->}(2.2,0.2)(6,3)

 % new vector 
\rput(6.35,1.5){$\mathcal{E}_N$}
\psline[linewidth=1.5 pt]{->}(6,0)(6,3)

 % new vector 
\rput(4,-0.3){$\hat{x}_N$}
\psline[linewidth=1.5 pt]{->}(2.2,0.2)(6,0)

 % new vector 
\psline[linewidth=1.5 pt](2.2,0.2)(6,0)


\end{pspicture}
\end{center}
We want to minimize $\abs{\left| \mathcal{E}_N\right|}$ to make the best approximation. $\mathcal{E}_N = x - \hat{x}_N$, hence $\abs{\left| \mathcal{E}_N\right|} = \abs{\left| x-\hat{x}_N\right|}$. Since $\mathcal{E} \perp W$, then we can use the inner product and the properties of orthogonality to solve for the coefficients. Since $W \in \R^{2N+1}$, we have $2N+1$ equations and $2N+1$ coefficients.

\begin{align*}
\langle \hat{x}_N, \Psi_\ell\rangle  &= \langle \sum \limits_{k=-N}^{N}\alpha_k e^{ik\omega_0 t} , \Psi_\ell\rangle  \\
\langle \hat{x}_N, \Psi_\ell\rangle  &= \sum \limits_{k=-N}^{N}\alpha_k \langle e^{ik\omega_0 t} , \Psi_\ell\rangle  \\
\langle \hat{x}_N, \Psi_\ell\rangle  &= \alpha_\ell \langle e^{i\ell\omega_0 t} , \Psi_\ell\rangle  \\
\langle \hat{x}_N, \Psi_\ell\rangle  &= \alpha_\ell \langle \Psi_\ell , \Psi_\ell\rangle  \\
\alpha_\ell &= \frac{\langle \hat{x}_N, \Psi_\ell\rangle }{\langle \Psi_\ell , \Psi_\ell\rangle } \\
\end{align*}

if $\hat{x}(t) = \sum \limits_{k=0}^{M-1}\alpha_k \Psi_k$, where $x$ is a $p$-periodic signal, which $\Psi_k$'s would you choose? Since we are taking a subset, the larger exponentials are better since they dominate. Otherwise, using $\{0,1,\dots,M\}\in\Z$ is not using very much. Pick the largest in magnitude FS coefficient, and then go down from there, but it doesn't have to be continguous. 


\begin{pspicture}(-5,-5)(5,5)
\rput(0.3,3.75){ $Im$ }
\psline{->}(0,-3.75)(0,3.75)
\rput(3.75,0.3){ $Re$ }
\psline{->}(-3.75,0)(3.75,0)
\pscircle(0,0){ 3 }
\pscircle(0,0){ 2 }
\pscircle(0,0){ 1 }
\rput(2.3,1){$e^{i\omega}-\alpha$}
\userline[linewidth=1.5 pt]{->}(1.500,0.000)(1.121,2.121)
\userline[linewidth=1.5 pt]{->}(1.500,0.000)(2.121,2.121)
\userline[linewidth=1.5 pt,linecolor=blue]{->}(0,0.000)(2.121,2.121){(x>0) ? 3 * cos( atan(-y/x) ) : -3 * cos( atan(-y/x) ) }{ (x>0) ? -3 * sin( atan(-y/x) ) : 3 * sin( atan(-y/x) )}
\userline[linewidth=1.5 pt,linestyle=dashed](-1.500,0.000)(2.121,2.121){x}{0}{x}{y}
\userline[linewidth=1.5 pt,linestyle=dashed](-1.500,0.000)(2.121,2.121){0}{y}{x}{y}
\rput(-0.75,-4.25){$1+\alpha$}
\rput(2.25,-4.25){$1-\alpha$}
\psline{<->}(-3,-4)(1.5,-4)
\psline{<->}(1.5,-4)(3,-4)
\psline[linestyle=dashed](3,-4.5)(3,0)
\psline[linestyle=dashed](-3,-4.5)(-3,0)
\psline[linestyle=dashed](1.5,-4.5)(1.5,0)
\end{pspicture}
